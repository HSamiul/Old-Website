\documentclass{article}

\usepackage{fancyhdr}
\usepackage{amsmath}
\pagestyle{fancy}
\fancyhf{}
\rhead{Grit}
\lhead{Mathematical Induction}
\rfoot{Page \thepage}
 
\begin{document}
 
\section{Introduction}
 
Visualize an infinitely long line of dominoes. If you knock the first one down yourself, you know the next one will fall, and so will the one after that.
Let's call any domino $k$. We can say if the $k=1$ domino falls, the $k=2$ domino falls. More generally, if the $k$'th domino falls, the $k+1$'th domino will as well. \\

\noindent
This idea is very powerful. We can use it to prove a rule true. Speaking arbitrarily, a rule can have $k$ cases. If we can prove the rule holds true for the $k+1$'th case, then its true for all cases. 

\section{Mathematical Induction}
A domino cannot fall by itself; it needs an initial push. This initial push is the \textbf{base case}. We must prove the base case true because we must have a $k$'th case to have a $k+1$'th case. \\

\noindent
Once we prove the base case true, we can try and prove the $k+1$'th case true, the \textbf{inductive step}. The $k+1$'th case is the infinitely long line of dominoes falling down one by one. \\

\noindent
We can apply this logic to something like a sequence. Let's write down the domino effect in mathematical terms.\\

\noindent
For a sequence $P(n)$ up to the $n$'th term, let the equation $S(n)$ represent the summation of $P(n)$'s elements:

\[
a_{1} + a_{2} + \dots + a_{n} = S(n)
\]

\noindent
Set $n = k$, where k is any real integer. \\

\noindent	
Base case: \\
\[
k = \text{some base value}
\]

\noindent
Assuming the equation $S(n)$ holds true to the sequence for $n = 1$, prove $S(n)$ holds true for $n = k + 1$. \\

\noindent
Inductive step:
\[
n = k + 1
\]
If the inductive step is successful, $S(n)$ is a valid rule.

\newpage
\section{Example}
Prove:
\[1^{3} + 2^{3} + \dots + n^{3} = \frac{n^{2}(n+1)^{2}}{4} \text{ for }  n \geq 0
\]

\noindent
Base case:
\begin{align}
k &= 0 \\
0^{3} &= \frac{0^{2}(n+1)^{2}}{4} \\
0 &= 0
\end{align}

\noindent
Inductive step: \\

\noindent
Adding the $k+1$'th term to $S(k)$ yields
\[
\frac{k^{2}(k+1)^{2}}{4} + (k+1)^{3}
\]
We must prove that 

\[
\frac{k^{2}(k+1)^{2}}{4} + (k+1)^{3} = \frac{(k+1)^{2}((k+1)+1)^{2}}{4} 
\]

\noindent
To explain the above step in plain English, we want to prove that $S(k)$ in terms of $k$ plus the $k+1$'th term is equal to $S(k+1)$ in terms of the $k+1$'th term. \\

\noindent
In other words, we know $S(k)$ is true in terms of $k$, but we need to prove its true for all cases, $k+1$.

\begin{align}
\frac{k^{2}(k+1)^{2}}{4} + (k+1)^{3} &= \frac{(k+1)^{2}((k+1)+1)^{2}}{4} \\[10pt]
k^{2}(k+1)^{2}+4(k+1)^{3} &= (k+1)^{2}(k+2)^{2} \\[10pt]
(k+1)\frac{k^{2}(k+1)^{2}+4(k+1)^{3}}{k+1} &= (k+1)^{2}(k+2)^{2} \\[10pt]
(k+1)(k+1)(k+2)^{2} &= (k+1)^{2}(k+2)^{2} \\[10pt]
\frac{(k+1)^{2}(k+2)^{2}}{4} &= \frac{(k+1)^{2}(k+2)^{2}}{4} 
\end{align}
\noindent
Note: In (6), we use the rational roots theorem to see that $k = -1$ is a root of the LHS, so we know we can divide by $k+1$ to create a simpler polynomial. \\

\noindent
S(n) is a valid summation formula  of the sequence P(n) for all $n\geq0$.

\section{Conclusion}
\noindent
The hardest part of mathematical induction is the algebra in the inductive step. You're trying to prove a formula $S(k)$ plus $k+1$ is equivalent to what we're trying to prove is true, $S(k+1)$. Once you've got that down, induction is just some algebraic gymnastics. \\

\noindent
Finally, mathematical induction isn't limited to proving domino-sequence like relationships. We can use it to prove many things, from summation formulas  to the maximum regions a square plane can be broken up into by n lines.

\section{References}
\begin{verbatim}
https://www.cs.cmu.edu/~adamchik/21-127/lectures/induction_1_print.pdf
\end{verbatim}

\begin{verbatim}
http://zimmer.csufresno.edu/~larryc/proofs/proofs.mathinduction.html
\end{verbatim}

\begin{verbatim}
https://brilliant.org/wiki/rational-root-theorem/
\end{verbatim}

\begin{verbatim}
my_algorithms_teacher.exe
\end{verbatim}
\end{document}
